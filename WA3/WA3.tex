\documentclass{article}
\usepackage{amssymb, amsmath, fullpage}
\usepackage{changepage}

\parskip 6pt
\parindent 0pt

\title{Writing Assignment 3 in \LaTeX}
\author{Alex Holland V00928553 }
\date{December 2, 2020}

\begin{document}

\maketitle

\bigskip

{\bf Question 1}\\ 
(a) show that $q,p_1,p_2$ are primes and $q|p_1p_2$, then $q=p_1$ or $q=p_2$.\\
(b) Suppose $q,p_1,p_2,p_3$ are primes and $q|p_1p_2p_3$. Prove that $q=p_1$ or $q=p_2$ or $q=p_3$.


\smallskip
(a)\\
We know that $q,p_1,p_2$ are primes and $q|p_1$ or $q|p_2$ as $q,p_1,p_2$ are primes. Because q and $p_1$ are primes then $q|p_1$ implies $q=p_1$. As well, q and $p_2$ are primes since $q|p_2$ implies $q=p_2$. Therefore, if $q,p_1,p_2$ are primes then $q|p_1p_2$ and so $q=p_1$. $\square$

\smallskip
(b)\\
$q,p_q,p_2,p_3$ are primes and $q|p_1p_2p_3$. Then $q|p_1p_2p_3$ implies $q|p_1$ or $q|p_2$ or $q|p_3$. if $q|p_1$ then $q=p_1$ since q and $p_1$ are primes. If $q|p_2$ the $q=p_2$ since q and $p_2$ are primes. If $q|p_3$ then $q=p_3$ since q and $p_3$ are primes. Therefore if $q,p_q,p_2,p_3$ are primes and $q|p_1p_2p_3$ then $q=p_1$ or $q=p_2$ or $q=p_3$. $\square$

\bigskip
{\bf Question 2} \\
Let $a,b,c \in \mathbb{N}$ be such that $c|a$ and $c|b$. Prove that $c|gcd(a,b)$.

\smallskip
We know that a and b are natural numbers. Let's let $d=gcd(a,b)$. Then there must exist integers $k_1$ and $k_2$ such that $d=ak_1+bk_2$. Because c divides a  implies $(c|a)$ $a=ck_1$ for some $k\in\mathbb{Z}$. Also c divides b implies $(c|b)$ $b=ck_2$ for some $k\in\mathbb{Z}$. Then
\begin{equation*} 
\begin{split}
    d & =ax+by\\
    & = ck_1x+ck_2y\\
    & = c(k_1x+k_2y)\\
\end{split}
\end{equation*}
Because $x,y,k_1,k_2 \in \mathbb{Z}$ then we have that $k_1x+k_2y \in \mathbb{Z}$ so $c(k_1x+k_2y)$. Therefore, $(c|d)$ which implies $c|gcd(a,b)$. $\square$

\bigskip
{\bf Question 3}\\
(a) Let $c \in \mathbb{N}$ and $m \in \mathbb{N}$. The least residue of n modulo m is the the unique integer among $0,1,...,m-1$ to which n is congruent modulo m. For $k \in \mathbb{N}$, explain why $k^2\equiv0(mod\;4)$ or $k^2\equiv1(mod\;4)$.\\
(b) Prove that no integer which is congruent to 4 modulo 4 can be written as a sum of two squares. That is, if $n\equiv3(mod\;4)$, then there are no integers $x$ and $y$ such that $n=x^2+y^2$.

\smallskip
(a)\\
We are given that $n \in \mathbb{Z}$ and $m \in \mathbb{N}$ The least residue of $n\;modulo\;m$ is the unique integer among $0,1,...m-1$ to which $n$ is congruent to $modulo\;m$. We can show $k^2\equiv0(mod\;4)$ or $k^2\equiv1(mod\;4)$ for $k \in \mathbb{N}$

\begin{equation*}
\begin{split}
    (4k+0)^2 & =16k^2\equiv0\;(mod\;4)\\
    (4k+1)^2 & =16k^2+8k+1\equiv1\;(mod\;4)\\
    (4k+2)^2 & =16k^2+16k+4\equiv0\;(mod\;4)\\
    (4k+3)^2 & =16k^2+24k+9\equiv1\;(mod\;4)\\
    (4k+4)^2 & =16k^2+32k+16\equiv0\;(mod\;4)\\
\end{split}    
\end{equation*}
We can determine from relationship of each equation that 0 and 1 are the only residues of $modulo\;4$ when we consider any integer $k$. Therefore for $k \in \mathbb{N}$, $k^2\equiv0(mod\;4)$ or $k^2\equiv1(mod\;4). $\square$

\smallskip
(b)\\
We need to prove that no integer which is congruent to $3\;modulo\;4$ can be written as a sum of two squares that is if $n=3\;(3\;mod\;4)$. Then there must be no integers $x$ and $y$ such that $n=x^2+y^2$. We can show the least residues of $square\;modulo\;3$ by the set of equations

\begin{equation*}
\begin{split}
    (3k)^2 & =9k^2 \\
    & \equiv0\;(mod\;3)\\
    (3k+1)^2 & = 9k^2+6k+1\\
    & \equiv 1\;(mod\;3)\\
    (3k-1)^2 & = 9k^2-6k+1\\
    & \equiv 0\;(mod\;3)\\
\end{split}
\end{equation*}
Hence, 0 and 1 are the only least residues of $modulo\;3$. So $x^2+y^2$ can take only the values 0, 1, and 2. Therefore, for any $n \in \mathbb{Z}$, $n\equiv3\;(mod\;4)$ such that $n=x^2+y^2$, thus there are no integers $x$ and $y$ which are congruent to $3\;modulo\;4$ that can be written as a sum of two squares. $\square$

\bigskip
{\bf Question 4}\\
Let $b>1$ be an integer, and $n=(d_k d_k_-_1...d_1d_0)_b$. Show that $(b-1)|n \Leftrightarrow d_0+d_1+d_2+...+d_k$.

\smallskip
Because $n=(d_k d_k_-_1...d_1d_0)_b$ then n is equivalent to $n=d_k \times b^k+d_k_-_1 \times b^k^-^1+...+d_1 \times b^1+d_0 \times b^0$. From this, we can see that
\begin{equation*}
\begin{split}
    b & \equiv 1\;(mod\;b-1)\\
    b^k & \equiv 1\;(mod\;b-1)\\
    d_k b^k & \equiv d_k\;(mod\;b-1)\\
    d_k_-_1 b^k^-^1 & \equiv d_k_-_1\;(mod\;b-1)\\
    d_1 b^1 & \equiv d_1\;(mod\;b-1)\\
    d_0 b^0 & \equiv d_0\;(mod\;b-1)\\
\end{split}
\end{equation*}
From this we can write $n$ as $n=d_k \times b^k+d_k_-_1 \times b^k^-^1+...+d_1 \times b^1+d_0 \times b^0 \equiv (d_k+d_k_-_1+d_1+d_0)(mod\;b-1)$. Therefore $(b-1)|n$ is equivalent to $d_0+d_1+d_2+...+d_k$. $\square$

\newpage

{\bf Question 5}\\
In this question we will give a proof that there are infinitely many primes that's similar to Euclid's proof. We'll do it in several steps. For a positive integer $n$, recall that $n \; factorial$ is the integer $n(n-1)(n-2)...1$.\\
(a) Suppose $k \in \mathbb{N}$ is such that $2 \leq k \leq n$. Explain why the remainder when $N=n!+1$ is divided by $k$ equals 1.\\
(b) Explain why part (a) implies that $N$ has a prime divisor greater that $n$.\\
(c) Explain why part (b) implies that there is no largest prime number.\\
(d) Explain why part (c) implies that there are infinitely many primes.

\smallskip
(a)\\
We can represent $N=n!+1$ as $n!=n(n-1)(n-2)...1$. Then $n!$ can be represented in terms of $n$, $N-n(n-1)(n-2)...1+1$. So, for every $k \in \mathbb{N}$, then $k|n!$. Therefore $N \equiv 1\;(mod\;k)$ for $1 \leq k \leq n$. $\square$

\smallskip
(b)\\
Part (a) implies that, for every $k \in \mathbb{N}$, such that $1 \leq k \leq n$. No $k$ can divide $N$. From this, no prime numbers from $1 \; to \; n$ can divide $N$, so either $N$ has a prime divisor greater than $n$, or $N$ is prime number, which then can not be divisible by any $k$, such that $1 \leq k \leq n$. $\square$

\smallskip
(c)\\
Suppose that $n$ is the largest prime number. By part (b), implies that $N=n!+1$ has a prime divisor that is greater than $n$ So, we get prime greater than n, for any $k \in \mathbb{Z}$ such that $1 \leq k \leq n$, which is a contradiction. There is no largest prime number because we always get a prime number that is greater than $n$. $\square$

\smallskip
(d)\\
It is determined by part (c) that there is no largest prime number, since for each prime number there exists a prime number that is larger. Therefore, this relation implies that there are infinitely many primes. $\square$


\end{document}